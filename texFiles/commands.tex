% !TeX root = ../proposal.tex
% بسته‌های فراخوانی‌شده و دستورهای استفاده‌شده:
% geometry
%\geometry{left=2cm, right=2cm, bottom=3cm, top=2cm}
% xcolor
% mdframed
% titlesec
%%%%%%%%%%%%%%%%%%%%%%%%%%%%%%%%%%%%%%%%%%%%%%%%%%%%%%
% بسته‌های ریاضی:
\usepackage{mathtools} % amsmath
\usepackage{amsfonts}
\usepackage{amssymb}
\usepackage{mleftright}
\RenewCommandCopy{\left}{\mleft}
\RenewCommandCopy{\right}{\mright}
%%%%%%%%%%%%%%%%%%%%%%%%%%%%%%%%%%%%%%%%%%%%%%%%%%%%%%
% بسته‌ای برای اصلاح اندازهٔ فونت:
\usepackage{fix-cm}
\usepackage{lmodern}
%%%%%%%%%%%%%%%%%%%%%%%%%%%%%%%%%%%%%%%%%%%%%%%%%%%%%%
% عکس و مسیر عکس‌ها:
\usepackage{graphicx}
\graphicspath{{./pictures/}}
%%%%%%%%%%%%%%%%%%%%%%%%%%%%%%%%%%%%%%%%%%%%%%%%%%%%%%
% بستهٔ مربوط به جدول‌ها و مشخصه‌های ستونی جدید:
\usepackage{array}
\newcolumntype{M}[1]{m{\dimexpr#1\textwidth-2\tabcolsep-2\arrayrulewidth}}
\newcolumntype{P}[1]{p{\dimexpr#1\textwidth-2\tabcolsep-2\arrayrulewidth}}
%%%%%%%%%%%%%%%%%%%%%%%%%%%%%%%%%%%%%%%%%%%%%%%%%%%%%%
% بسته‌ای برای ادغام چند ردیف از جدول:
\usepackage{multirow}
%%%%%%%%%%%%%%%%%%%%%%%%%%%%%%%%%%%%%%%%%%%%%%%%%%%%%%
% بسته‌ای برای مدیریت کپشن‌ها:
\usepackage{caption}
\captionsetup{font=small, aboveskip=2pt, width=0.85\linewidth}
%%%%%%%%%%%%%%%%%%%%%%%%%%%%%%%%%%%%%%%%%%%%%%%%%%%%%%
% بسته‌ای برای تنظیم فاصلهٔ خطوط:
\usepackage{setspace}
%%%%%%%%%%%%%%%%%%%%%%%%%%%%%%%%%%%%%%%%%%%%%%%%%%%%%%
% بسته‌ای برای مدیریت ارجاع به منابع:
\makeatletter
\ifznu@useRefNum
	\usepackage{cite}
	\bibliographystyle{unsrt-fa}
\else
	\usepackage{natbib}
	\bibliographystyle{chicago-fa}
\fi
\makeatother
%%%%%%%%%%%%%%%%%%%%%%%%%%%%%%%%%%%%%%%%%%%%%%%%%%%%%%
% بسته‌های جدید خود را اینجا فراخوانی کنید:
\usepackage{fontawesome5}
%%%%%%%%%%%%%%%%%%%%%%%%%%%%%%%%%%%%%%%%%%%%%%%%%%%%%%
% پرش و رنگی‌کردن لینک‌ها:
\makeatletter
\ifznu@print
	\usepackage[hidelinks]{hyperref}
\else
	\usepackage[pagebackref]{hyperref}
\fi
\makeatother
%%%%%%%%%%%%%%%%%%%%%%%%%%%%%%%%%%%%%%%%%%%%%%%%%%%%%%
% بستهٔ زی‌پرشین و فونت‌ها:
\usepackage[%
	perpagefootnote=on%
]{xepersian}
\settextfont[Path={./fonts/},%
	Scale=1.1,%
	ItalicFont={IRLotusICEE_Iranic.ttf},%
	BoldFont={IRLotusICEE_Bold.ttf},%
	BoldItalicFont={IRLotusICEE_BoldIranic.ttf},%
]{IRLotusICEE.ttf}
\makeatletter
\ifznu@digitFa
	\setmathdigitfont[Path={./fonts/},%
		Scale=1.1,%
		ItalicFont={IRLotusICEE_Iranic.ttf},%
		BoldFont={IRLotusICEE_Bold.ttf},%
		BoldItalicFont={IRLotusICEE_BoldIranic.ttf},%
	]{IRLotusICEE.ttf}
\fi
\makeatother
%%%%%%%%%%%%%%%%%%%%%%%%%%%%%%%%%%%%%%%%%%%%%%%%%%%%%%
% تعریف برخی حروف ایستاده در محیط ریاضی:
\newcommand{\diff}{\ensuremath{\,\mathrm d}}
\newcommand{\I}{\ensuremath{\mathrm i}}
\newcommand{\E}{\ensuremath{\mathrm e}}