% !TeX root = ../proposal.tex
\section{تعریف مسئله و پیشینۀ پژوهش}

در ابتدا، انتظار می‌ره که با سیستم حروف‌چینی
\lr{\LaTeX}
آشنایی کافی را کسب کرده باشین.

برای کار با این قالب، به کامنت‌های گذاشته‌شده در هر بخش توجه کنین. همچنین یک ویدئو در این خصوص آماده
شده است که می‌تونین از لینک‌های ذیل دریافت و مشاهده کنین:
\begin{itemize}
	\item \href{https://www.aparat.com/Bahman_Mirzakhani/playlists}{آپارات}
	\item\href{https://www.youtube.com/channel/UCzZoLZNsM9utoqycWoG7f5A/featured}{\lr{YouTube}}
	\item
	همچنین، آخرین نسخهٔ قالب، در
	\href{https://github.com/bahman-mirzakhani/znu-enj-proposal.git}{\lr{GitHub}}
	قرار داده می‌شود.
\end{itemize}
حین استفاده، اگه متوجه مشکلی در قالب شدین، می‌تونین از طریق مورد اول لیست
زیر گزارش و از موارد دیگه برای سؤالات مربوط به لیتک استفاده کنین.
\begin{itemize}
	\begin{latinitems}
		\item[\faMailBulk]
		\href{mailto:bmirzakhani.en@gmail.com}%
			{bmirzakhani.en@gmail.com}
		\item[\faInternetExplorer]
		\href{http://qa.parsilatex.com/}{http://qa.parsilatex.com}
		\item[\faInternetExplorer]
		\href{https://tex.stackexchange.com/}{https://tex.stackexchange.com}
	\end{latinitems}
\end{itemize}

بسته‌های ذکرشده در ابتدای فایل
\texttt{commands.tex}،
فراخوانی شده‌اند و نیازی به فراخوانی مجدد آن‌ها نیست و می‌تونین از دستورات اون بسته‌ها،
در صورت نیاز، استفاده کنین. اگه به بسته‌های جدیدی نیاز داشتین، در قسمت مشخص‌شدهٔ
فایل
\texttt{commands.tex}،
فراخوانی کنین. برای مثال، بستۀ
\texttt{fontawesome5}،
جهت استفاده در آیکون‌های لیست قبلی، فراخوانی شده است. بعد از پاک‌کردن این متن،
اون بسته رو هم پاک کنین؛ چون احتمالاً بهش نیاز ندارین.

\begin{description}
\item[توصیه:] \itshape
اگه به بسته‌ای نیاز ندارین، فراخوانی نکنین.
\end{description}

نیازی به نصب فونت خاصی نیست و فونت‌های مورد نیاز در پوشۀ
\texttt{fonts}
قرار داده و استفاده شده‌اند.

\section{نوآوری و تفاوت کار پیشنهادی با تحقیقات قبلی}

\section{اهداف}

\section{فرضیه‌ها یا سؤالات پژوهش}

\section{روش اجرای پژوهش}

\section{%
	جامعه آماری، تعداد نمونه و روش نمونه‌گیری (در صورت لزوم)}

\section{روش تجزیه و تحلیل اطلاعات}

\section{فهرست منابع و مآخذ}
منابع به‌کمک
\lr{\textsc{Bib}\TeX}
ایجاد و در فایل
\verb|References.bib|
در پوشهٔ
\verb|texFiles|
ذخیره می‌شوند. برای منابع فارسی، بایستی مدخلِ
\begin{LTR}
	\verb|language = {persian},|
\end{LTR}\noindent
نیز اضافه گردد.

\subsection{روش عددی}
در این روش، بایستی آپشن
\lr{refNumber}
برای کلاس نوشتار، فعال شده باشد و برای ارجاع‌دهی، از دستور
\verb|\cite{label}|
استفاده می‌شود. برای مثال، ارجاع به مراجع
\cite[فصل 4]{abtahi1388latex}،
\cite{oommen2002}
و
\cite{knuth1984texbook}
به تنهایی و همچنین با هم:
\cite{abtahi1388latex,oommen2002,knuth1984texbook}.
در روش عددی، نیازی به مدخلِ
\texttt{authorfa}
نیست و می‌توانید ننویسید.

این قسمت نیاز به اجرای
\lr{\textsc{Bib}\TeX}
دارد که ترتیب اجرا به این صورت است:
\begin{latin}
	\begin{verbatim}
		xelatex
		bibtex
		xelatex
		xelatex
	\end{verbatim}
\end{latin}

\subsection{روش نویسنده-سال}
در این روش، بایستی آپشن
\lr{refNumber}
برای کلاس نوشتار، غیرفعال شده باشد و برای منابع لاتین، اسم فارسی نویسندگان را در مدخلِ
\begin{LTR}
	\verb|authorfa = {Family, Name and Family, Name},|
\end{LTR}\noindent
وارد کنید. دقت کنید که دقیقاً با فرمت ذکرشده وارد شوند. به فایل
\verb|References.bib|
مراجعه کنید.

برای ارجاع در وسط جمله، از دستور
\verb|\citet{label}|
و در انتهای جمله، از دستور
\verb|\citep{label}|
استفاده می‌شود. اگر قصد نوشتن اسامی نویسندگان خارجی را در پاورقی دارید،
بعد از ارجاع، از دستور پاورقی استفاده کنید؛ یعنی:
\begin{LTR}
	\verb|\citet{label}\LTRfootnote{\citeauthor*{label}}|
\end{LTR}

این قسمت نیاز به اجرای
\lr{\textsc{Bib}\TeX8}
دارد. دستور مربوطه، در فایل
\verb|execution.txt|
داخل پوشهٔ
\verb|texFiles|
نوشته شده است. یک دستور جدید در ویرایشگر خود تعریف و استفاده کنید. ترتیب اجرا به این صورت است:
\begin{latin}
	\begin{verbatim}
		xelatex
		bibtex8 -W -c cp1256fa
		xelatex
		xelatex
	\end{verbatim}
\end{latin}
\begin{RefBox}
	\bibliography{./texFiles/References}\vskip0pt
\end{RefBox}