% E-mail: bmirzakhani.en@gmail.com
% Telegram: @Bahman_Mirzakhani
% Version: 1.4a
% اگر مقطع دکتری هستید، آپشن msc را  غیرفعال کنید.
% اگر قصد دارید اعداد در محیط ریاضی، انگلیسی باشند، آپشن digitFa را غیرفعال کنید.
% اگر روش ارجاع‌دهی به‌صورت عددی است، آپشن refNumber را فعال کنید.
% اگر قصد چاپ پیشنهاده دارید، آپشن print را فعال کنید.
\documentclass[%
	msc,%
	digitFa,%
	refNumber,%
%	print%
]{./texFiles/znuEnjP}
% بسته‌ها و تنظیمات مورد نیاز:
% !TeX root = ../proposal.tex
% بسته‌های فراخوانی‌شده و دستورهای استفاده‌شده:
% geometry
%\geometry{left=2cm, right=2cm, bottom=3cm, top=2cm}
% xcolor
% mdframed
% titlesec
%%%%%%%%%%%%%%%%%%%%%%%%%%%%%%%%%%%%%%%%%%%%%%%%%%%%%%
% بسته‌های ریاضی:
\usepackage{mathtools} % amsmath
\usepackage{amsfonts}
\usepackage{amssymb}
\usepackage{mleftright}
\RenewCommandCopy{\left}{\mleft}
\RenewCommandCopy{\right}{\mright}
%%%%%%%%%%%%%%%%%%%%%%%%%%%%%%%%%%%%%%%%%%%%%%%%%%%%%%
% بسته‌ای برای اصلاح اندازهٔ فونت:
\usepackage{fix-cm}
\usepackage{lmodern}
%%%%%%%%%%%%%%%%%%%%%%%%%%%%%%%%%%%%%%%%%%%%%%%%%%%%%%
% عکس و مسیر عکس‌ها:
\usepackage{graphicx}
\graphicspath{{./pictures/}}
%%%%%%%%%%%%%%%%%%%%%%%%%%%%%%%%%%%%%%%%%%%%%%%%%%%%%%
% بستهٔ مربوط به جدول‌ها و مشخصه‌های ستونی جدید:
\usepackage{array}
\newcolumntype{M}[1]{m{\dimexpr#1\textwidth-2\tabcolsep-2\arrayrulewidth}}
\newcolumntype{P}[1]{p{\dimexpr#1\textwidth-2\tabcolsep-2\arrayrulewidth}}
%%%%%%%%%%%%%%%%%%%%%%%%%%%%%%%%%%%%%%%%%%%%%%%%%%%%%%
% بسته‌ای برای ادغام چند ردیف از جدول:
\usepackage{multirow}
%%%%%%%%%%%%%%%%%%%%%%%%%%%%%%%%%%%%%%%%%%%%%%%%%%%%%%
% بسته‌ای برای مدیریت کپشن‌ها:
\usepackage{caption}
\captionsetup{font=small, aboveskip=2pt, width=0.85\linewidth}
%%%%%%%%%%%%%%%%%%%%%%%%%%%%%%%%%%%%%%%%%%%%%%%%%%%%%%
% بسته‌ای برای تنظیم فاصلهٔ خطوط:
\usepackage{setspace}
%%%%%%%%%%%%%%%%%%%%%%%%%%%%%%%%%%%%%%%%%%%%%%%%%%%%%%
% بسته‌ای برای مدیریت ارجاع به منابع:
\makeatletter
\ifznu@useRefNum
	\usepackage{cite}
	\bibliographystyle{unsrt-fa}
\else
	\usepackage{natbib}
	\bibliographystyle{chicago-fa}
\fi
\makeatother
%%%%%%%%%%%%%%%%%%%%%%%%%%%%%%%%%%%%%%%%%%%%%%%%%%%%%%
% بسته‌های جدید خود را اینجا فراخوانی کنید:
\usepackage{fontawesome5}
%%%%%%%%%%%%%%%%%%%%%%%%%%%%%%%%%%%%%%%%%%%%%%%%%%%%%%
% پرش و رنگی‌کردن لینک‌ها:
\makeatletter
\ifznu@print
	\usepackage[hidelinks]{hyperref}
\else
	\usepackage[pagebackref]{hyperref}
\fi
\makeatother
%%%%%%%%%%%%%%%%%%%%%%%%%%%%%%%%%%%%%%%%%%%%%%%%%%%%%%
% بستهٔ زی‌پرشین و فونت‌ها:
\usepackage[%
	perpagefootnote=on%
]{xepersian}
\settextfont[Path={./fonts/},%
	Scale=1.1,%
	ItalicFont={IRLotusICEE_Iranic.ttf},%
	BoldFont={IRLotusICEE_Bold.ttf},%
	BoldItalicFont={IRLotusICEE_BoldIranic.ttf},%
]{IRLotusICEE.ttf}
\makeatletter
\ifznu@digitFa
	\setmathdigitfont[Path={./fonts/},%
		Scale=1.1,%
		ItalicFont={IRLotusICEE_Iranic.ttf},%
		BoldFont={IRLotusICEE_Bold.ttf},%
		BoldItalicFont={IRLotusICEE_BoldIranic.ttf},%
	]{IRLotusICEE.ttf}
\fi
\makeatother
%%%%%%%%%%%%%%%%%%%%%%%%%%%%%%%%%%%%%%%%%%%%%%%%%%%%%%
% تعریف برخی حروف ایستاده در محیط ریاضی:
\newcommand{\diff}{\ensuremath{\,\mathrm d}}
\newcommand{\I}{\ensuremath{\mathrm i}}
\newcommand{\E}{\ensuremath{\mathrm e}}
\setstretch{1.4}
\begin{document}
% مشخصات جدول کارشناس تحصیلات تکمیلی (در صورت نیاز):
	% !TeX root = ../proposal.tex
% مشخصات جدول کارشناس تحصیلات تکمیلی
% شماره ثبت:
\sabt{}

% کد رهگیری:
\rahgiri{}

% تاریخ تصویب:
\tasvib{} %yyyy/mm/dd

\logo
% مشخصات پایان‌نامه و دانشجو:
	% !TeX root = ../proposal.tex
% مشخصات پایان‌نامه
% عنوان فارسی پایان‌نامه:
\titleFa{عنوان فارسی پایان‌نامه}

% عنوان انگلیسی پایان‌نامه:
\titleEn{English title of thesis}

% واژگان کلیدی فارسی:
\keywordsFa{کلیدواژۀ اول، کلیدواژۀ دوم، کلیدواژۀ سوم}

% واژگان کلیدی انگلیسی:
\keywordsEn{first keyword, second keyword, third keyword}
%%%%%%%%%%%%%%%%%%%%%%%%%%%%%%%%%%%%%%%%%%%%%%%%%%%%%%%%%%%%%%%%%%%
% مشخصات دانشجو
% نام کامل دانشجو:
\fullName{نام دانشجو}

% سال ورود:
\yearIn{xxxx}

% شماره دانشجویی:
\studentNumber{xxxxxxxx}

% دانشکده:
\department{نام دانشکده}

% رشته:
\major{رشتۀ دانشجو}

% گرایش:
\field{گرایش دانشجو}

% آدرس:
\studentAddress{آدرس دانشجو}

% تلفن ثابت:
\lpNumber{(024) ww xx yy zz}

% همراه:
\cpNumber{09ww xxx yy zz}
% مشخصات اساتید راهنما:
	% !TeX root = ../proposal.tex
% مشخصات استاد راهنما(ی اول)
% نام و نام‌خانوادگی:
\firstSupervisorFullName{استاد راهنما(ی اول)}

% محل خدمت:
\firstSupervisorOffice{دانشگاه زنجان}

% تخصص اصلی:
\firstSupervisorSpecialty{تخصص استاد راهنما}

% آخرین مدرک:
\firstSupervisorDegree{دکتری}

% رتبۀ دانشگاهی:
\firstSupervisorGrade{استادیار}

% میزان مشارکت:
\firstSupervisorParticipation{50}

% آدرس:
\firstSupervisorAddress{آدرس استاد راهنما}

% تلفن محل کار:
\firstSupervisorPhone{(024) ww xx yy zz}
%%%%%%%%%%%%%%%%%%%%%%%%%%%%%%%%%%%%%%%%%%%%%%%%%%%%%%%%%%%%%%%%%%%
% اگر استاد راهنمای دوم ندارید، به کدهای زیر دست نزنید.
% اگر استاد راهنمای دوم دارید، کافیست نام و نام‌خانوادگی را در خط زیر وارد و فعال نمایید:
%\secSupervisorFullName{استاد راهنمای دوم}

% محل خدمت:
\secSupervisorOffice{دانشگاه زنجان}

% تخصص اصلی:
\secSupervisorSpecialty{تخصص استاد راهنما}

% آخرین مدرک:
\secSupervisorDegree{دکتری}

% رتبۀ دانشگاهی:
\secSupervisorGrade{استادیار}

% میزان مشارکت:
\secSupervisorParticipation{50}

% آدرس:
\secSupervisorAddress{آدرس استاد راهنما}

% تلفن محل کار:
\secSupervisorPhone{(024) ww xx yy zz}
% مشخصات اساتید داور و مشاور:
	% !TeX root = ../proposal.tex
% مشخصات اساتید داور:
\masterRefereeA{داور اول}

\masterRefereeB{داور دوم}

% اگر مقطع دکتری هستید، داور سوم را در دستور زیر وارد نمایید:
%\phdRefereeC{داور سوم}
%%%%%%%%%%%%%%%%%%%%%%%%%%%%%%%%%%%%%%%%%%%%%%%%%%%%%%%%%%%%%%%%%%%
% اگر استاد مشاور ندارید، به کدهای زیر دست نزنید.
% اگر استاد مشاور دارید، کافیست نام و نام‌خانوادگی را در خط زیر وارد و فعال نمایید:
%\firstAdvisorFullName{استاد مشاور (اول)}

% محل خدمت:
\firstAdvisorOffice{دانشگاه زنجان}

% تخصص اصلی:
\firstAdvisorSpecialty{تخصص استاد مشاور}

% آخرین مدرک:
\firstaAdvisorDegree{دکتری}

% رتبۀ دانشگاهی:
\firstAdvisorGrade{استادیار}

% میزان مشارکت:
\firstAdvisorParticipation{50}

% آدرس:
\firstAdvisorAddress{آدرس استاد مشاور}

% تلفن محل کار:
\firstAdvisorPhone{(024) ww xx yy zz}
%%%%%%%%%%%%%%%%%%%%%%%%%%%%%%%%%%%%%%%%%%%%%%%%%%%%%%%%%%%%%%%%%%%
% اگر استاد مشاور دوم ندارید، به کدهای زیر دست نزنید.
% اگر استاد مشاور دوم دارید، کافیست نام و نام‌خانوادگی را در خط زیر وارد و فعال نمایید:
%\secAdvisorFullName{استاد مشاور دوم}

% محل خدمت:
\secAdvisorOffice{دانشگاه زنجان}

% تخصص اصلی:
\secAdvisorSpecialty{تخصص استاد مشاور}

% آخرین مدرک:
\secAdvisorDegree{دکتری}

% رتبۀ دانشگاهی:
\secAdvisorGrade{استادیار}

% میزان مشارکت:
\secAdvisorParticipation{50}

% آدرس:
\secAdvisorAddress{آدرس استاد مشاور}

% تلفن محل کار:
\secAdvisorPhone{(024) ww xx yy zz}
% نوع پژوهش، نوع اجرا و …:
	% !TeX root = ../proposal.tex
% با توجه به نوع پژوهش و اجرا، دستور checked را متناسب با مورد خود استفاده کنید
% نوع پژوهش: بنیادی، کاربردی، توسعه‌ای
\researchType{%
	بنیادی
	$ \square $
	\hspace{\stretch{1}}
	کاربردی
	$ \square $
	\hspace{\stretch{1}}
	توسعه‌ای
	$ \checked $}
% نوع اجرا: نظری، نظری-عملی، عملی
\executionType{%
	نظری
	$ \checked $
	\hspace{\stretch{1}}
	نظری-عملی
	$ \square $
	\hspace{\stretch{1}}
	عملی
	$ \square $}
% تعداد واحدهای پایان‌نامه:
\courseUnits{6}
% زمان اخذ واحد:
\unitDate{1398/05/01}
% مدت اجرا:
\executionTime{12 ماه}

% چاپ اطلاعات واردشده
% به این قسمت دست نزنید
% اطلاعات پایان‌نامه:
\thesisInfo
% اطلاعات دانشجو:
\studentInfo
% اطلاعات استاد راهنمای اول:
\supervisorInfo
% اطلاعات استاد راهنمای دوم:
\secSupervisorInfo
% اطلاعات استاد مشاور اول:
\advisorInfo
% اطلاعات استاد مشاور دوم:
\secAdvisorInfo
% اطلاعات همین فایل:
\timeTable
% متن پیشنهاده در فایل زیر نوشته می‌شود:
	% !TeX root = ../proposal.tex
\section{تعریف مسئله و پیشینۀ پژوهش}

در ابتدا، انتظار می‌ره که با سیستم حروف‌چینی
\lr{\LaTeX}
آشنایی کافی را کسب کرده باشین.

برای کار با این قالب، به کامنت‌های گذاشته‌شده در هر بخش توجه کنین. همچنین یک ویدئو در این خصوص آماده
شده است که می‌تونین از لینک‌های ذیل دریافت و مشاهده کنین:
\begin{itemize}
	\item \href{https://www.aparat.com/Bahman_Mirzakhani/playlists}{آپارات}
	\item\href{https://www.youtube.com/channel/UCzZoLZNsM9utoqycWoG7f5A/featured}{\lr{YouTube}}
	\item
	همچنین، آخرین نسخهٔ قالب، در
	\href{https://github.com/bahman-mirzakhani/znu-enj-proposal.git}{\lr{GitHub}}
	قرار داده می‌شود.
\end{itemize}
حین استفاده، اگه متوجه مشکلی در قالب شدین، می‌تونین از طریق مورد اول لیست
زیر گزارش و از موارد دیگه برای سؤالات مربوط به لیتک استفاده کنین.
\begin{itemize}
	\begin{latinitems}
		\item[\faMailBulk]
		\href{mailto:bmirzakhani.en@gmail.com}%
			{bmirzakhani.en@gmail.com}
		\item[\faInternetExplorer]
		\href{http://qa.parsilatex.com/}{http://qa.parsilatex.com}
		\item[\faInternetExplorer]
		\href{https://tex.stackexchange.com/}{https://tex.stackexchange.com}
	\end{latinitems}
\end{itemize}

بسته‌های ذکرشده در ابتدای فایل
\texttt{commands.tex}،
فراخوانی شده‌اند و نیازی به فراخوانی مجدد آن‌ها نیست و می‌تونین از دستورات اون بسته‌ها،
در صورت نیاز، استفاده کنین. اگه به بسته‌های جدیدی نیاز داشتین، در قسمت مشخص‌شدهٔ
فایل
\texttt{commands.tex}،
فراخوانی کنین. برای مثال، بستۀ
\texttt{fontawesome5}،
جهت استفاده در آیکون‌های لیست قبلی، فراخوانی شده است. بعد از پاک‌کردن این متن،
اون بسته رو هم پاک کنین؛ چون احتمالاً بهش نیاز ندارین.

\begin{description}
\item[توصیه:] \itshape
اگه به بسته‌ای نیاز ندارین، فراخوانی نکنین.
\end{description}

نیازی به نصب فونت خاصی نیست و فونت‌های مورد نیاز در پوشۀ
\texttt{fonts}
قرار داده و استفاده شده‌اند.

\section{نوآوری و تفاوت کار پیشنهادی با تحقیقات قبلی}

\section{اهداف}

\section{فرضیه‌ها یا سؤالات پژوهش}

\section{روش اجرای پژوهش}

\section{%
	جامعه آماری، تعداد نمونه و روش نمونه‌گیری (در صورت لزوم)}

\section{روش تجزیه و تحلیل اطلاعات}

\section{فهرست منابع و مآخذ}
منابع به‌کمک
\lr{\textsc{Bib}\TeX}
ایجاد و در فایل
\verb|References.bib|
در پوشهٔ
\verb|texFiles|
ذخیره می‌شوند. برای منابع فارسی، بایستی مدخلِ
\begin{LTR}
	\verb|language = {persian},|
\end{LTR}\noindent
نیز اضافه گردد.

\subsection{روش عددی}
در این روش، بایستی آپشن
\lr{refNumber}
برای کلاس نوشتار، فعال شده باشد و برای ارجاع‌دهی، از دستور
\verb|\cite{label}|
استفاده می‌شود. برای مثال، ارجاع به مراجع
\cite[فصل 4]{abtahi1388latex}،
\cite{oommen2002}
و
\cite{knuth1984texbook}
به تنهایی و همچنین با هم:
\cite{abtahi1388latex,oommen2002,knuth1984texbook}.
در روش عددی، نیازی به مدخلِ
\texttt{authorfa}
نیست و می‌توانید ننویسید.

این قسمت نیاز به اجرای
\lr{\textsc{Bib}\TeX}
دارد که ترتیب اجرا به این صورت است:
\begin{latin}
	\begin{verbatim}
		xelatex
		bibtex
		xelatex
		xelatex
	\end{verbatim}
\end{latin}

\subsection{روش نویسنده-سال}
در این روش، بایستی آپشن
\lr{refNumber}
برای کلاس نوشتار، غیرفعال شده باشد و برای منابع لاتین، اسم فارسی نویسندگان را در مدخلِ
\begin{LTR}
	\verb|authorfa = {Family, Name and Family, Name},|
\end{LTR}\noindent
وارد کنید. دقت کنید که دقیقاً با فرمت ذکرشده وارد شوند. به فایل
\verb|References.bib|
مراجعه کنید.

برای ارجاع در وسط جمله، از دستور
\verb|\citet{label}|
و در انتهای جمله، از دستور
\verb|\citep{label}|
استفاده می‌شود. اگر قصد نوشتن اسامی نویسندگان خارجی را در پاورقی دارید،
بعد از ارجاع، از دستور پاورقی استفاده کنید؛ یعنی:
\begin{LTR}
	\verb|\citet{label}\LTRfootnote{\citeauthor*{label}}|
\end{LTR}

این قسمت نیاز به اجرای
\lr{\textsc{Bib}\TeX8}
دارد. دستور مربوطه، در فایل
\verb|execution.txt|
داخل پوشهٔ
\verb|texFiles|
نوشته شده است. یک دستور جدید در ویرایشگر خود تعریف و استفاده کنید. ترتیب اجرا به این صورت است:
\begin{latin}
	\begin{verbatim}
		xelatex
		bibtex8 -W -c cp1256fa
		xelatex
		xelatex
	\end{verbatim}
\end{latin}
\begin{RefBox}
	\bibliography{./texFiles/References}\vskip0pt
\end{RefBox}
% زمان‌بندی مراحل اجرای پژوهش:
	% !TeX root = ../proposal.tex
\section{زمان‌بندی مراحل اجرای پژوهش (از تصویب تا دفاع)}
\begin{center}
	\begin{tabular}{|M{.6}|>{\centering\arraybackslash}M{.2}|}
		\hline
		\multicolumn{1}{|M{.6}|}{\textbf{مرحله}} &
		\text{\textbf{مدت‌زمان (ماه)}} \\ \hline
		مرحله اول & 4
		\\\hline
		مرحله دوم & 5
		\\\hline
		مرحلهٔ سوم & 1
		\\\hline
	\end{tabular}
\end{center}
% هزینه‌ها:
	% !TeX root = ../proposal.tex
\section{جمع کل هزینه‌ها}
% اگر برای هر ردیف مواردی وجود دارد، می‌توانید داخل آکلادهای مربوطه اضافه نمایید.
% مواد و وسایل:
\materials{(برای مثال)}
\materialCost{1000}

% پرسنلی:
\personal{}
\personalCost{500}

% مسافرت:
\travel{}
\travelCost{}

% متفرقه:
\other{(تایپ، تکثیر و تهیه کتاب)}
\otherCost{}

% جمع کل:
\totalCost{1500}

% این پیشنهاده بخشی از طرح‌های
\inOut{%
	داخل دانشگاه
	$ \checked $
	خارج دانشگاه
	$ \square $
%	است
}
% عنوان اصلی طرح تحقیقاتی مرتبط:
\mainTitle{}

% نام مجری:
\executive{}

% دانشکده:
\faculty{}

% تاریخ تصویب نهایی:
\finalDate{}

% بودجه مصوب:
\approved{}

\costs
	
% اگر مقطع کارشناسی‌ارشد هستید و قصد ندارید محل امضای داور سوم چاپ گردد، خط زیر را فعال کنید:

%	\csname znu@thirdRefereetrue\endcsname

% چاپ جدول محل امضای دانشجو، اساتید راهنما، مشاور و داور:
	\signature
% چاپ صورت‌جلسه:
%	\clearpage
	\proceedings
\end{document}